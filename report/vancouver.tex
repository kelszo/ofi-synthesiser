\usepackage[style=numeric,
    backend=biber,
    sorting=none,
    url=false,
    isbn=false,
    terseinits=true,
    giveninits=true,
    minnames=6,
    maxnames=6]{biblatex}

% Remove unwanted punctuations
\renewcommand*{\revsdnamepunct}{}
\renewcommand*{\finentrypunct}{}
\renewcommand*{\bibpagespunct}{}

% Dot instead av brackets in references
\DeclareFieldFormat{labelnumberwidth}{\mkbibbold{#1\adddot}}

% Lastname followed by initials format
\DeclareNameAlias{sortname}{family-given}
\DeclareNameAlias{default}{family-given}
\renewcommand*{\revsdnamepunct}{}

\DeclareSourcemap{%
    \maps[datatype=bibtex]{
        % Journal abbreviations
        \map[overwrite]{
            \step[fieldsource=shortjournal]
            \step[fieldset=journaltitle,origfieldval]
        }
    }
}

% remove in
\renewbibmacro{in:}{}
% remove pp
\DeclareFieldFormat*{pages}{#1}
% reformat doi
\DeclareFieldFormat*{doi}{\url{https://doi.org/#1}}
%remove quotation marks around title
\DeclareFieldFormat*{title}{#1}


\DeclareFieldFormat{journaltitle}{\mkbibemph{#1}\isdot}

% Provide three letter month names
\newcommand*{\shortmonth}[1]{
    \ifthenelse{\NOT\equal{#1}{}}{
        \ifcase#1\relax
        \or Jan
        \or Feb
        \or Mar
        \or Apr
        \or May
        \or Jun
        \or Jul
        \or Aug
        \or Sep
        \or Oct
        \or Nov
        \or Dec
        \fi
    }
}

\DeclareFieldFormat*{number}{\mkbibparens{#1}}

\DeclareFieldFormat*{date}{\thefield{year}}

% Code adapted from biblatex-nejm package

\renewbibmacro*{volume+number+eid}{
    \printfield{volume}%
    \setunit{}%
    \printfield{number}%
    \addcolon%
    \printfield{eid}%
}

\renewbibmacro*{issue+date}{
    \usebibmacro{date}
}

\renewbibmacro*{journal+issuetitle}{
    \usebibmacro{journal}%
    \iffieldundef{series}%
    \adddot%
    {}
    {\newunit%
        \printfield{series}}%
    \setunit{\addspace}%
    \usebibmacro{issue+date}%
    \setunit{\addsemicolon}%
    \usebibmacro{volume+number+eid}%
    \usebibmacro{issue}%
    \newunit}

% compress page numbers. E.g. XYZ-XAB -> XYZ-AB
\DeclareFieldFormat{postnote}{\mkcomprange[{\mkpageprefix[pagination]}]{#1}}
\DeclareFieldFormat{pages}{\mkcomprange{#1}}

% Compress ranges where lower limit > 100
\setcounter{mincomprange}{100}

% Don't compress beyond the fourth digit
\setcounter{maxcomprange}{1000}

% Display compressed upper limit with at least two digits,
% unless leading digit is zero
\setcounter{mincompwidth}{10}