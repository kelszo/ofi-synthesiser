\documentclass[12pt, a4paper]{article}
\usepackage[english]{babel}

% margins
\usepackage[a4paper, margin=2.5cm]{geometry}

% include pdf
\usepackage[final]{pdfpages}

\usepackage{graphics}

% Special/larger tables
\usepackage{longtable}
\usepackage{tabularx}

% Centered X
\newcolumntype{Y}{>{\centering\arraybackslash}X}
% Centered p
\newcolumntype{P}[1]{>{\centering\arraybackslash}p{#1}}

% quotes
\usepackage{csquotes}

% Numbers and SI units
\usepackage{siunitx}
\sisetup{
  group-four-digits = true,
  group-separator = {,}
}


% Acronyms
\usepackage[nopostdot,style=super,nonumberlist,toc,nogroupskip,acronym]{glossaries}

% Double spacing
\usepackage{setspace}

% appendix
\usepackage[page, title]{appendix}

% tables
\usepackage{booktabs}
\usepackage{array} % centering + width fix
\usepackage{multirow} % use multi rows/columns

\usepackage{caption} 
\captionsetup{labelfont=bf,labelsep=period,font=small,justification=raggedright,singlelinecheck=false} 

% Times New Roman
%\usepackage{newtxtext,newtxmath}

% Page numbering right
\usepackage{fancyhdr}
\pagestyle{fancy}
% Clear the header and footer
\fancyhead{}
\fancyfoot{}
\renewcommand{\headrulewidth}{0pt}
% Set the right side of the footer to be the page number
\fancyfoot[R]{\thepage}

% Fix section titles size/format
\usepackage{titlesec}
\titleformat{\section}
  {\normalfont\fontsize{16}{19.2}\bfseries}
  {\thesection}
  {1em}
  {}

\titleformat{\subsection}
  {\normalfont\fontsize{14}{17}\bfseries}
  {\thesubsection}
  {1em}
  {}

  \titleformat{\subsubsection}
  {\normalfont\fontsize{12}{14}\itshape} 
  {\thesubsubsection}
  {1em}
  {}

\titlespacing{\section}{0pt}{\parskip}{-\parskip}
\titlespacing{\subsection}{0pt}{0.75\parskip}{-\parskip}
\titlespacing{\subsubsection}{0pt}{0.5\parskip}{-\parskip}

% Remove section numbers
\setcounter{secnumdepth}{0}

% fix line breaks urls
\PassOptionsToPackage{hyphens}{url}
\usepackage[hidelinks,pdfusetitle]{hyperref}

\usepackage{url}
\usepackage{cleveref}

% Set paragraph spacing
\setlength{\parindent}{0pt}
\setlength{\parskip}{1em}
\def\ni{\noindent}

%% These two lines are needed to get the correct paper size
%% in TeX Live 2016
\let\pdfpageheight\paperheight
\let\pdfpagewidth\paperwidth

\makenoidxglossaries

\newacronym{auc}{AUC}{area under the receiver operating characteristic curve}
\newacronym{mam}{M\&M}{morbidity and mortality}
\newacronym{ml}{ML}{machine learning}
\newacronym{ofi}{OFI}{opportunity for improvement}
\newacronym{qi}{QI}{quality improvement}
\newacronym{gan}{GAN}{generative adversarial network}
\newacronym{iss}{ISS}{injury severity score}
\newacronym{swetrau}{SweTrau}{Swedish Trauma Registry}
\newacronym{ci}{CI}{confidence interval}
\newacronym{ctgan}{CTGAN}{conditional tabular generative adversarial network}
\newacronym{tvae}{TVAE}{triplet based variational autoencoder}
\newacronym{vae}{VAE}{variational variable autoencoder}
\newacronym{sd}{SD}{standard deviation}

% Acronym style
\newglossarystyle{csuper}{%
    \setglossarystyle{super}%
    \renewcommand{\glossentry}[2]{%
        \glsentryitem{##1}\glstarget{##1}{\glossentryname{##1}} &
        \Glossentrydesc{##1}\glspostdescription\space ##2\tabularnewline
    }
    \setlength{\glsdescwidth}{0.75\linewidth}
}

% References
\usepackage[style=numeric-comp,
    backend=biber,
    sorting=none,
    url=false,
    isbn=false,
    terseinits=true,
    giveninits=true,
    minnames=6,
    maxnames=6]{biblatex}

% Remove unwanted punctuations
\renewcommand*{\revsdnamepunct}{}
\renewcommand*{\finentrypunct}{}
\renewcommand*{\bibpagespunct}{}

% Remove 'in:' string
\renewbibmacro{in:}{}

% Provide three letter month names
\newcommand*{\shortmonth}[1]{
    \ifthenelse{\NOT\equal{#1}{}}{
        \ifcase#1\relax
        \or Jan
        \or Feb
        \or Mar
        \or Apr
        \or May
        \or Jun
        \or Jul
        \or Aug
        \or Sep
        \or Oct
        \or Nov
        \or Dec
        \fi
    }
}

% Family name first
\DeclareNameAlias{default}{family-given}

% Remove unwanted formatting
\DeclareFieldFormat*{title}{#1}
\DeclareFieldFormat*{journaltitle}{#1}
\DeclareFieldFormat*{labelnumberwidth}{#1\adddot}
\DeclareFieldFormat*{pages}{:\mkcomprange{#1}}
\DeclareFieldFormat*{url}{Available from: \url{#1}}
\DeclareFieldFormat*[article]{issue}{(#1)}
\DeclareFieldFormat*{date}{
    \thefield{year}%
    \shortmonth{\thefield{month}}\addspace
    \thefield{day}\isdot
}

\DeclareSourcemap{
    \maps[datatype=bibtex]{
        \map[overwrite=true]{
            \step[fieldsource=journaltitle, match=\regexp{(\.)}, replace={}]
            \step[fieldsource=journaltitle, match=\regexp{(\$)}, replace={.}]
        }
    }
}

\renewbibmacro*{volume+number+eid}{
    \printfield{volume}%
    \printfield{issue}%
    \printfield{eid}%
}
%Order year;volume:page
\renewbibmacro*{issue+date}{
    \setunit*{\addspace}%
    \usebibmacro{date}%
    \newunit%
}

\renewbibmacro*{journal+issuetitle}{
    \usebibmacro{journal}%
    \addperiod%
    \iffieldundef{series}
    {}
    {\newunit
        \printfield{series}}%
    \setunit*{\addspace}%
    \usebibmacro{issue+date}%
    \setunit*{\addsemicolon\addspace}%
    \usebibmacro{volume+number+eid}%
    \usebibmacro{issue}%
    \newunit}


\DeclareFieldFormat*{urldate}{
    [cited \thefield{urlyear}%
            \shortmonth{\thefield{urlmonth}}\addspace%
            \printfield{urlday}]
}

\DeclareBibliographyDriver{database}{%
    \printfield{title} [Internet]%
    \setunit*{\addperiod\space}%
    \printlist{location}%
    \setunit*{\addcolon\space}% 
    \printnames{author}%
    \setunit*{\addsemicolon\space}%
    \printfield{year}%
    \setunit*{ - .}%
    \printurldate{}%
    \setunit*{\addperiod\space}%
    \printfield{url}%
    \finentry}

\DeclareBibliographyDriver{report}{%
    \usebibmacro{author}%
    \setunit*{\addperiod\space}%
    \printfield{title}%
    \setunit*{\addperiod\space}%
    \printlist{location}%
    \setunit*{\addcolon\space}% 
    \printlist{institution}%
    \setunit*{\addsemicolon\space}%
    \printfield{year}%
    \setunit*{\addperiod\space}%
    \finentry}

\renewrobustcmd{\mkbibbrackets}{\mkbibparens}

% \let\bibopenbracket\bibopenparen
%\let\bibclosebracket\bibcloseparen
 %imports biblatex 
\addbibresource{main.bib}

\author{Kelvin Szolnoky}
\title{Using Synthetic Data to Improve Opportunity for Improvement Prediction Model Performance- A Registry Based Study in Trauma Patient Care}

% Titles
% Using Synthetic Data to Improve Opportunity for Improvement Prediction Model Performance
% Improving the Performance of Models to Predict Opportunities for Improvement in Trauma Care Using Synthetic Data
% Using Synthetic Data to Improve the Performance of Models to Predict Opportunities for Improvement in Trauma Care



\begin{document}

\begin{titlepage}
	\includepdf[pages=-,pagecommand={},fitpaper=true,]{title_page.pdf}
\end{titlepage}
\setstretch{1}
\fontsize{11}{13}\selectfont

\addcontentsline{toc}{section}{Abstract}

\textbf{Bättre modeller för att predicera förbättringsmöjligheter i traumavård med hjälp av syntetisk data - en registerbaserad studie} \\
\textit{Bakgrund:} . \textit{Syfte:} . \textit{Material och Metoder:} . \textit{Resultat:} . \textit{Slutsats:}
\vfill

\textbf{Using Synthetic Data to Improve Opportunity for Improvement Prediction Model Performance - A Registry Based Study } \\
\textit{Introduction:} Trauma is a leading cause of mortality and morbidity. One way to ensure high-quality care, and lower the mortality and morbidity, is by identifying opportunities for improvement (OFI). One way to identify OFIs is by using machine learning (ML). Recently, using generative adversarial networks (GAN) and variational autoencoders (VAE) to synthesise data and improve ML models performance has been proposed. \textit{Aims:} To investigate if GANs and VAEs can be used for improving ML OFI prediction models in trauma patient care. \textit{Material and Methods:} Trauma patients admitted to Karolinska University Hospital between 2014 and 2021 screened for OFI and over the age of 15 were included. Eight ML models were developed and trained on baseline, synthetic data generated by a GAN and VAE, and both. Performance was measured using area under the receiver operating characteristic curve (AUC). \textit{Results:} The best performing model on the baseline data, random forest (AUC: 0.792 [0.787, 0.797]), significantly outperformed itself being trained on GAN (AUC: 0.565 [0.551, 0.579]) and VAE (AUC: 0.637 [0.615, 0.659]) synthesised data. Using GAN and VAE synthesised data combined with baseline data did not improve performance compared with baseline, with a performance of 0.777 (0.772, 0.782) and 0.776 (0.771, 0.781) AUC respectively. The average AUC for the ML models on the baseline data was 0.753, compared to 0.555 and 0.566 for the GAN and VAE models' data respectively. \textit{Conclusions:} The results of the current study indicates that current methods for synthesising data does not improve ML models performance of OFI prediction.
\vfill

\textit{Keywords}: Machine learning; Trauma; Data synthesis; Medical audit; Trauma care quality improvement; Prediction

\newpage

\normalsize

\glsaddall
\printnoidxglossary[type=acronym,style=csuper]

\newpage

\setstretch{1.45}

\section{Introduction}
Trauma, the clinical entity composed by physical injury and the body's associated response~\cite{gerdin_risk_2015}, is
a significant contributor to mortality and morbidity for people between the ages of 10 and 49 globally, and it is the
leading cause of death among young people in Sweden~\cite{roth_global_2018, vos_global_2020, sos_death_2021}. The
average age of the trauma population is low, and two-thirds of this group have no previous history of
co-morbidity~\cite{brattstrom_socio-economic_2015}. The estimated age standardised disability-adjusted life years
(DALY) for the trauma population is 247.6 million~\cite{haagsma_global_2016}. In comparison, the most common cause of
mortality globally, ischaemic heart disease, has a DALY of 170.3 million
years~\cite{wang_global_2021,roth_global_2018}. As a result, it is crucial to provide high-quality trauma care to
reduce mortality and morbidity rates for this population, which has a long-life expectancy.

One way to enhance trauma patient care is through the implementation of \acrfull{qi} programs, as recommended by the
World Health Organization (WHO)~\cite{world_health_organization_guidelines_2009}. These \acrshort{qi} programs have a
base in the three concepts for quality assurance suggested by Donabedian - structure, process, and
outcomes~\cite{donabedian_effectiveness_1996}. \Acrfull{mam} conferences are a crucial component of these programs and
aim to identify \acrfull{ofi} in patient care~\cite{santana_development_2014}. \acrshort{mam} conferences are conducted
by representatives from all disciplines and professions involved in trauma care, during which the treatment provided to
an individual patient is evaluated and compared to the optimal treatment that should have been provided. Regularly
conducting these reviews is linked to a reduction in complication rates, hospital stay time, and preventable deaths,
thereby providing high-quality trauma care~\cite{stelfox_evidence_2011, mcdermott_trauma_1994}. The conferences aim to
address all pillars suggested by Donabedian for quality assurance, however, due to their nature, \acrshort{mam}
conferences are highly resource-intensive.

An \acrshort{ofi}, identified by an \acrshort{mam} conference, refers to a specific deficit in patient care and often
occurs in initial care, including airway management, fluid resuscitation, haemorrhage control, and chest injury
management~\cite{world_health_organization_guidelines_2009,roy_learning_2017,oreilly_opportunities_2013,sanddal_analysis_2011}.
The prevalence of \acrshortpl{ofi} for the trauma population as a whole has not been well researched, however, at
Karolinska University Hospital the prevalence for \acrshortpl{ofi} in the trauma population is at
7\%~\cite{attergrim_predicting_2023}. In other words, if all trauma patients were brought up at a \acrshort{mam},
fourteen in fifteen cases would not result in an \acrshort{ofi} being found.

Filtering cases with an audit filter system before reviewing by an \acrshort{mam} conference provides a possibility to
reduce false positive cases and thus reduce the costs of the \acrshort{qi} programme as a whole. Therefore, a more
advanced \acrshort{qi} technique suggested by World Health Organization is the application of audit filters that use
electronic health record systems to monitor predefined variables to flag patient cases with potential
\acrfull{ofi}~\cite{world_health_organization_guidelines_2009}.

However, the performance of audit filter systems is inconsistent and has been linked to high rates of false positives.
Depending on the context, the frequency of false positives can range from 24\% to
80\%~\cite{attergrim_predicting_2023,sanddal_analysis_2011,roy_learning_2017,ghorbani_analysis_2018}. It has been
suggested to use the calculated Probability of Survival score (PS) and Trauma Score and Injury Severity Score (TRISS)
as audit filters, however, TRISS and PS have a poor track record in detecting a substantial number of preventable
deaths and miss important \acrshort{ofi}~\cite{heim_survival_2016}.

A newer approach proposed by Attergrim et al.~\cite{attergrim_predicting_2023} recommends replacing audit filter
systems with supervised \acrfull{ml} models. \acrshort{ml} is a subfield of artificial intelligence that involves the
use of algorithms and statistical models to enable computers to improve their performance on a specific task based on
data input~\cite{greener_guide_2022}. In medicine, \acrshort{ml} can be used to analyse large amounts of data and
identify patterns and relationships that may be difficult or impossible for humans to
discern~\cite{greener_guide_2022}. \acrshort{ml} in trauma has existed for some time and most studies investigate the
application of \acrshort{ml} models in prediction mortality and complications~\cite{zhang_machine_2022}.

Attergrim and colleagues used \acrshort{ml} algorithms to analyse a trauma registry including data about demographics,
clinical parameters, injury type and severity, treatment outcomes, and occurring \acrshortpl{ofi}. By analysing this
data, \acrshort{ml} algorithms could identify patterns and relationships that were not immediately apparent to humans,
and thus also the currently used audit filters. The models could successfully find opportunities for improvement in
trauma patient care. This is the first time that \acrshort{ml} models have been used to predict \acrshort{ofi} in
trauma care.~\cite{attergrim_predicting_2023}

These newer \acrshort{ml} models, developed by Attergrim and colleagues, exhibit a significantly improved performance
compared to the audit filter system currently used at Karolinska University Hospital in Stockholm, Sweden. When
maintaining the same sensitivity, the false positive rate was reduced from 68\% to 53\% with the use of the same
data~\cite{attergrim_predicting_2023}.

The models investigated by Attergrim et al.~\cite{attergrim_predicting_2023} include traditional \acrshort{ml} methods:
logistic regression, random forest~\cite{breiman_random_2001}, decision trees, support vector
machines~\cite{cortes_support-vector_1995}, as well as newer boosting models: XGBoost~\cite{chen_xgboost_2016},
LightGBM~\cite{ke_lightgbm_2017}, and CatBoost~\cite{prokhorenkova_catboost_2018}. These are general purpose
\acrshort{ml} models that were specifically trained on registry data to be able to identify \acrshort{ofi} in trauma
patient care. The overall best performing model was LightGBM.

Despite the improved performance, the models still have a problem in that 50\% of patient cases are inaccurately marked
as \acrshort{ofi} positive. \acrshort{ml} methods generally require substantial amounts of data to perform
effectively~\cite{piccialli_survey_2021}. There are several ways to enhance the performance of \acrshort{ml} models,
and the most basic and well-known is simply increasing the amount of data~\cite{greener_guide_2022}. Until now, the
only option for doing so was by manually gathering more data and annotating it. This process is slow, and in some
cases, not feasible due to a shortage of data.

Another way of increasing data is by generating new synthetic data from already existing data, this is called synthetic
data generation. \acrshort{ml} can be used for synthetic data generation by training generative models on real-world
datasets to create new, artificial data with similar statistical properties and characteristics as the original
data.~\cite{chen_synthetic_2021}

In the medical field, synthetic data generation can be used to create datasets that mimic real patient data, without
compromising patient privacy~\cite{liu_ppgan_2019}. For example, a computer might be trained on a dataset of patient
health records, and used to generate new records that have similar characteristics, such as age, gender, and medical
conditions. These new records can be used to study and analyse medical data, without the risk of exposing sensitive
patient information~\cite{liu_ppgan_2019}. They may also be used to increase performance of other \acrshort{ml}
models~\cite{chen_synthetic_2021}. These synthetic datasets can, in theory, be distributed and increase the
availability of medical datasets for researchers to further open and reproducible research.

Recently, the use of \acrfullpl{gan} to generate synthetic data has become popular in image
analysis~\cite{pavan_kumar_generative_2021}. A \acrshort{gan} is a type of artificial neural network consisting of two
main components: a generator and a discriminator. The generator is trained to produce new data samples that resemble
the input data, while the discriminator is trained to differentiate between the generated samples and the real
data.~\cite{pavan_kumar_generative_2021}

The generator and discriminator are trained together in an adversarial manner. The generator aims to produce samples
that the discriminator cannot identify as fake, while the discriminator strives to accurately identify the fake samples
generated by the generator. This process allows the generator to learn to produce samples that resemble the input data,
and the discriminator to learn to distinguish between real and generated data. As a result, the generator continually
improves, producing data that more closely resembles the source and becomes increasingly difficult for the
discriminator to identify as synthetic data.~\cite{pavan_kumar_generative_2021}

An example of this as the \acrfull{ctgan}, a specialised type of GAN that is specifically designed for generating
synthetic tabular data. The ``conditional'' refers to the fact that the generator network takes as input not only a
random noise vector, as in a normal \acrshort{gan}, but also a set of ``conditioning'' variables that specify some
characteristics of the desired output. Such conditions could be for example the range that a variable can be generated
in.~\cite{xu_modeling_2019}

Another type of generative model in artificial neural networks is the \acrfull{vae}. It is a simpler and more compact
model that learns the probabilistic representation of a dataset, called the latent space. This representation is a
compressed form of the input data that captures its most important features and variations. The \acrshort{vae} consists
of two parts: an encoder and a decoder. The encoder maps the input data to a probability distribution in the latent
space, while the decoder maps a sample from that distribution back to the original data space. By training the
\acrshort{vae}, we learn a compact representation of the input data that highlights its important features and
variations.~\cite{kingma_auto-encoding_2013}

The most used \acrshort{vae} is the \acrfull{tvae}. Shortly, \acrshort{tvae} is trained to learn three different latent
representations: one for the input data, one for a ``positive'' example that is similar to the input, and one for a
``negative'' example that is dissimilar to the input. By training the TVAE on this triplet of latent representations,
it is possible to learn more robust and discriminative features than with a traditional VAE.~\cite{ishfaq_tvae_2018}

The quality of the generated data from these methods often varies and is dependent on several factors, such as the
quality and quantity of the baseline data~\cite{karras_training_2020}. The major advantage of having a method for
generating high-quality synthetic data is the ability to enlarge small datasets and train \acrshort{ml} models in cases
where it was not feasible before, or did not produce satisfactory results due to a lack of data. This has been
previously explored in medical image analysis contexts where it has also been shown to improve cross-site
generalisation~\cite{sanaat_robust-deep_2022, bashyam_deep_2022}.

However, the use of \acrshortpl{gan} and \acrshortpl{vae} for structured tabular data is relatively uncommon and
largely unexplored in trauma. There are few studies that investigate the possibility of generating synthetic tabular
data in a trauma context~\cite{hernandez_synthetic_2022}.

\subsection{Aim}
The aim of this study is to evaluate the effectiveness of using \acrshortpl{gan} and \acrshortpl{vae} for generating
synthetic tabular data in a trauma setting, and to explore the potential of this approach for improving \acrshort{ofi}
prediction models.

\section{Materials and Methods}
We conducted a retrospective study using registry data from Karolinska University Hospital, comparing the performance
of supervised machine learning models that were trained with synthetic data and those that were not, by analysing all
trauma patients in the trauma registry and quality database from 2014 to 2021.

The code used in this study is available online at \url{https://codeberg.org/kelszo/ofi-synthesiser} under the
\textit{GNU Affero General Public License v3.0} license.

\subsection{Study Population and Setting}
Karolinska University Hospital in Solna, Sweden is a level 1 equivalent trauma center that manage approximately 1500
acute trauma patients annually. The hospital reports all patients admitted with an \acrfull{iss} score of more than 9,
as well as all patients admitted with trauma team activation regardless of \acrshort{iss}, to an internal trauma
registry. Karolinska University Hospital's trauma registry is linked to the national \acrfull{swetrau}~\cite{swetrau}.
The \acrshort{swetrau} registry includes data on patient parameters such as vital signs, injuries, interventions, and
various checkpoint times. The registry is compliant with the Utstein template for uniform reporting of data following
major trauma.~\cite{ringdal_utstein_2008}.

Parallel to the registry, Karolinska University Hospital maintains an internal \acrshort{qi} registry known here as the
quality database. This database contains information related to the \acrshort{qi} program and \acrshort{mam}
conferences, including audit filter results and identified \acrshortpl{ofi} and their recommended corrective actions. A
table containing all currently used audit filters at the hospital can be found in the appendix
(\Cref{tab:auditfilters}).

The Karolinska University Hospital conducts multidisciplinary conferences to review the outcomes of trauma patients,
involving all healthcare professionals involved in their initial care. This includes surgery, anaesthesia,
orthopaedics, intensive care, radiology, and nursing. \acrshort{ofi} are identified through a consensus decision, and
appropriate corrective actions are suggested. Mortality cases are immediately escalated to the mortality conference,
while morbidity cases include escalating levels of reviews. The mortality conferences conclude whether a patient's
mortality was preventable or possibly preventable. The review process for morbidity cases was improved and formalised
during the study period, with audit filters applied to the trauma quality database and individual reviews by
specialised nurses added in 2017.

Between 2014 and 2017, a specialised trauma nurse selectively reviewed trauma patients to identify possible
\acrshortpl{ofi} to escalate to a \acrshort{mam} conference. Following 2017, a specialised nurse registered data to the
trauma registry while at the same time looking for possible audit filter violations (including a possible manual
violation flag) and applying them to the trauma quality database. If a violation was detected, the patient was reviewed
again by two specialised nurses. If the second review could not exclude a possible \acrshort{ofi}, the patient case was
finally escalated to a morbidity conference for a final verdict.

All patients screened for \acrshort{ofi} between 2014 and 2021 were included. Patients under the age of 15 were
excluded due to differing clinical pathway.

\subsection{Predictor Variables}
All variables included in the trauma registry and the revised Utstein template for major trauma were used as predictors
for both \acrshort{ofi} classification and data synthesising \acrshort{ml} models. The predictors included both
categorical and continuous values, such as blood pressure, interventions, injury mechanisms and intentions, care length
and level, and time points. The trauma registry includes data for the entire patient care, including pre-hospital
setting and treatment, in-hospital care, and outcome in the form of Glasgow Outcome Scale and mortality. For a complete
list of predictors and their definitions, see the revised Utstein template for major trauma~\cite{ringdal_utstein_2008}
or \Cref{tab:predictors} in the appendix.

\subsection{Outcome variable}
The outcome was defined as the verdict from the \acrshort{mam} conference with the binary levels of ``Yes - At least
one \acrshort{ofi} identified'' or ``No - No \acrshort{ofi} identified''. Preventable or possible preventable deaths
were considered as an \acrshort{ofi}.

\subsection{Model Development and Data preprocessing}
All model development and data preprocessing was carried out in Python 3.8~\cite{python_2009}. The model development
process was divided into two parts: the development of the classification models and the development of the
synthesising models. For detailed information regarding implementation specifics, versions, and packages, see the
project source code.

\subsubsection*{Data preprocessing}
A preprocessor was developed to preprocess both the training and test data without any data leakage. This was achieved
by transforming continuous features using Yeo-Johnson's power transformation~\cite{yeo_new_2000} and encoding nominal
features using one-hot encoding. Ordinal features remained unchanged. Continuous features were imputed using the mean
of the feature, ordinal features using the most frequent value, and nominal features using an ``unknown'' category. An
exception was made for blood pressure and respiratory rate in the pre-hospital and emergency department, which were
imputed using the mean value of the revised trauma score category~\cite{ringdal_utstein_2008} if it existed. The
preprocessor was trained on the training data and then applied to the test data, ensuring no data leakage between the
test and training sets.

\subsubsection*{Classification models}
As suggested by Attergrim et al.~\cite{attergrim_predicting_2023}, the following traditional methods were
implemented using scikit-learn~\cite{pedregosa_scikit_2011}: logistic regression, random forest, ensemble
of decision trees, support vector machine, and $k$-nearest neighbor. The following boosting methods were
implemented using their respective Python implementations: XGBoost~\cite{chen_xgboost_2016},
LightGBM~\cite{ke_lightgbm_2017}, and CatBoost~\cite{prokhorenkova_catboost_2018}. Hyperparameter tuning
was performed on all models using the first resampled training data, with a 5-fold cross-validation. The suggested
parameters found in the documentation were optimised using Tree-structured Parzen
Estimator~\cite{bergstra_algorithms_2011} implemented by the Optuna framework~\cite{optuna_2019} for a total of
1 hour per model.

\subsubsection*{Synthesising models}
The \acrfull{ctgan}~\cite{xu_modeling_2019} and \acrfull{tvae}~\cite{ishfaq_tvae_2018} models were implemented with the
Synthetic Data Vault framework~\cite{patki_sdv_2016} in Python. The batch size was increased to 1000 samples to
incorporate a higher number of minority cases in a single batch. The number of epochs trained was increased to 600.
The standard parameters provided were otherwise used.

\subsection{Statistical Analysis}
All statistical analyses were performed using Python. The model analyses were conducted by splitting the data into an
80\%-20\% train-test and repeating the process 100 times using random resampling without replacement. The estimated
95\% \acrfull{ci} using $t$-distribution was calculated for \acrfull{auc} using all resamples.

The synthesising models were trained on the raw training data from each resample. The synthesising models generated an
output dataset with an equal size to the input training data. If the number of generated \acrshort{ofi} positive cases
was less than in the input, 1000 new batches were generated and the \acrshort{ofi} positive cases were added to the
original generated dataset. The classification models were trained on either the pre-processed training data from the
resample, the pre-processed output of the synthesising models, or a combination of both (\Cref{fig:modelflowchart}).
During each training cycle, the classification models were tested on the pre-processed test data from the train-test
split, and the \acrshort{auc} was calculated. This resulted in each classification model being evaluated five times for
each resample: using the raw training data, the \acrshort{ctgan} synthetic data, a combination of the raw training data
and \acrshort{ctgan} synthetic data, the \acrshort{tvae} synthetic data, and a combination of the raw training data and
\acrshort{tvae} synthetic data. Each data input mentioned is referred to as a data group.

\begin{figure}
	\centering
	\includegraphics[width=\textwidth]{figures/model_flowchart.pdf}
	\caption{\textbf{Flowchart describing the process of evaluating the synthesising models.}
		Process is replicated for each classification model and resample. Each arrow to ``Train Classification Model''
		indicates an individual and separated training/evaluation cycle from the other data inputs.\\
		*Synthetic training data output by a data synthesising model.\\
		\textit{Definition of abbreviations:} CTGAN = Conditional tabular generative adversarial network;
		TVAE = Triplet based variational autoencoder.}%
	\label{fig:modelflowchart}
\end{figure}

The performance of the models was assessed on the test split from each resample and compared using \acrshort{auc}. The
performance of a classification model was compared both to models within the same data group and to models in different
data groups.

Descriptive statistics, including mean, \acrfull{sd}, median, minimum, and maximum for continuous and ordinal features,
and prevalence for categorical features, were calculated. The missing prevalence for all features were calculated.
Descriptive statistics were computed for both the baseline and synthetic datasets and grouped by: overall,
\acrshort{ofi} positive, and \acrshort{ofi} negative. Descriptive statistics for patient demographic and clinical
characteristics variables were chosen based on previously chosen variables that are important for \acrshort{ofi}
detection~\cite{attergrim_predicting_2023}.

In order to assess the quality of the synthetic datasets, the performance of \acrshort{ofi} classification models
trained on the synthetic datasets were compared against models trained on baseline data. Additionally, distributions of
descriptive statistics between the synthetic datasets and baseline data were analyzed to provide insights into the
similarity or dissimilarity of the datasets. Furthermore, the feasibility of variables in the synthetic datasets were
assessed, particularly in the context of medical data. Synthetic data that included data where variables, such as
respiratory rate, heart rate, and Glascow Coma Scale, exhibited values that are biologically/clinically implausible
were assessed as poor quality synthetic data.

\subsection{Ethical Considerations}
The group under investigation in this study, trauma patients, can be considered a vulnerable population due to the
severe outcomes that often result from major trauma, such as disability and death. Thus, special consideration must be
given to the ethics of conducting this study. It is important to note that only anonymous registry data was used in
this study, and no identifiable personal data was accessible by the authors or the \acrshort{ml} models, this is one of
the important measurements taken to reduce possible ethical complications.

In terms of autonomy and non-maleficence, no interventions were made that could potentially harm the patients. The
registry on which this study was based, \acrshort{swetrau}, operates under the assumption of presumed consent, meaning
that if a patient does not explicitly opt out of being included in the registry, it is assumed that they have consented
to participate. While this study poses almost no risk of harm to the study population, the absence of sensitive
personal data also reduces the risk of stigmatising a particular group.

The study aligns with the principle of beneficence, as its purpose is to improve clinical models for detecting
\acrshortpl{ofi} and improve patient care in the long run. The study provides valuable insights into methods to improve
the performance of \acrshort{ml} models in this area.

The ethical considerations associated with synthesising personal data warrant further discussion. If it were possible
to generate 100\% anonymous synthesised data, would the same ethical principles apply? Since the synthesised data does
not represent actual patients, it is difficult to consider autonomy and justice in this context. However, conducting
epidemiological studies on randomly generated or simulated data is typically done without much ethical consideration.
There must be a consensus on these ethical considerations before synthesising models can produce highly accurate
synthetic data.

In conclusion, the benefits of this study, as measured against the four principles of Beauchamp and Childress
(autonomy, non-maleficence, descence, and justice), outweigh the potential risks. The study was approved by the
Stockholm Research Ethics Review Board (permits 2021-02541 and 2021-03531).

\section{Results}
\subsection{Patient Characteristics}

Between 2014 and 2021, a population of \num{12107} trauma patients was identified, of which \num{6314} were excluded
due to not being screened for an \acrshort{ofi} by either audit filters or individual review. None of the patients were
excluded for being under the age of 15. Therefore, a total of \num{5793} patients were deemed eligible and included in
the study. Among these cases, \num{5453} patients had no \acrshort{ofi}, while \num{340} patients were diagnosed with
at least one \acrshort{ofi}. A summarised graph of the inclusion and exclusion criteria, as well as the process to an
\acrshort{ofi}, can be seen in \Cref{fig:flowchart}.

\begin{figure}
	\centering
	\includegraphics[width=0.6\textwidth]{figures/flowchart.pdf}
	\caption{\textbf{Flowchart describing the exclusions made and including the process to determine a case with \acrshort{ofi}.} The amount of patient cases brought up at a morbidity conference is unknown in the current dataset.\\
		*Defined at 30-days post trauma.\\
		\textit{Definition of abbreviations:} OFI = opportunity for improvement.}%
	\label{fig:flowchart}
\end{figure}

Sample characteristics are summarised in \Cref{tab:tableone}. The sample as a whole had a mean age of 46
(\acrshort{sd}: 21), while the \acrshort{ofi} and \acrshort{ofi}-negative subgroups had a mean age of 50
(\acrshort{sd}: 22) and 45 (\acrshort{sd}: 21), respectively. The majority of patients were male ($n = $ \num{3509},
61\%), and the overall group had a mortality rate of 6\% ($n = $ \num{331}). The \acrshort{ofi} group had a higher mean
\acrshort{iss} (18 [\acrshort{sd}: 10] vs 10 [\acrshort{sd}: 11]) and were admitted to the intensive care unit more
frequently (30\% [$n = $ \num{286}] vs 14\% [$n = $ \num{761}]). Data for emergency procedures were less missing in the
\acrshort{ofi} subgroup, with 54\% ($n = $ 182) of records missing data compared to 82\% ($n = $ \num{4452}) in the
\acrshort{ofi}-negative group. Furthermore, the ``other'' emergency procedure was more common in the \acrshort{ofi}
group (26\% [$n = $ \num{88}] vs 13\% [$n = $ \num{699}]).

\begin{table}[p]
	\centering
	\renewcommand{\arraystretch}{0.9}
	\caption{\textbf{Selected patient demographic and clinical characteristics}.
		Subdivided by \acrshort{ofi} outcome for baseline data.}%
	\label{tab:tableone}
	\scalebox{0.7}{
		\begin{tabular}{lccc}
			\toprule
			                                              & \textbf{Overall}  & \textbf{No OFI}   & \textbf{OFI}     \\
			\midrule
			                                              & $n=5793$          & $n=5453$          & $n=340$          \\
			\textbf{Age}                                  &                   &                   &                  \\
			\hspace{3mm}Mean (SD)                         & 46 (21)           & 45 (21)           & 50 (22)          \\
			\hspace{3mm}Median [Min, Max]                 & 43 [15, 100]      & 43 [15, 100]      & 50 [15, 97]      \\
			\hspace{3mm}Missing                           & 651 (11\%)        & 617 (11\%)        & 34 (10\%)        \\
			\textbf{Gender}                               &                   &                   &                  \\
			\hspace{3mm}Male                              & 3509 (61\%)       & 3293 (60\%)       & 216 (64\%)       \\
			\hspace{3mm}Female                            & 1633 (28\%)       & 1543 (28\%)       & 90 (26\%)        \\
			\hspace{3mm}Missing                           & 651 (11\%)        & 617 (11\%)        & 34 (10\%)        \\
			\textbf{Dead at 30 days}                      &                   &                   &                  \\
			\hspace{3mm}No                                & 4809 (83\%)       & 4523 (83\%)       & 286 (84\%)       \\
			\hspace{3mm}Yes                               & 331 (6\%)         & 311 (6\%)         & 20 (6\%)         \\
			\hspace{3mm}Missing                           & 653 (11\%)        & 619 (11\%)        & 34 (10\%)        \\
			\textbf{Highest level of care}                &                   &                   &                  \\
			\hspace{3mm}Intensive care unit               & 863 (15\%)        & 761 (14\%)        & 102 (30\%)       \\
			\hspace{3mm}General ward                      & 2002 (35\%)       & 1933 (35\%)       & 69 (20\%)        \\
			\hspace{3mm}Surgical ward                     & 945 (16\%)        & 852 (16\%)        & 93 (27\%)        \\
			\hspace{3mm}ED                                & 1120 (19\%)       & 1106 (20\%)       & 14 (4\%)         \\
			\hspace{3mm}Specialist/Intermediate ward      & 212 (4\%)         & 184 (3\%)         & 28 (8\%)         \\
			\hspace{3mm}Missing                           & 651 (11\%)        & 617 (11\%)        & 34 (10\%)        \\
			\textbf{Injury severity score}                &                   &                   &                  \\
			\hspace{3mm}Mean (SD)                         & 10 (11)           & 10 (11)           & 18 (10)          \\
			\hspace{3mm}Median [Min, Max]                 & 8 [0, 75]         & 5 [0, 75]         & 17 [0, 59]       \\
			\hspace{3mm}Missing                           & 653 (11\%)        & 619 (11\%)        & 34 (10\%)        \\
			\textbf{ED Respiratory Rate}                  &                   &                   &                  \\
			\hspace{3mm}Mean (SD)                         & 23 (20)           & 23 (20)           & 24 (21)          \\
			\hspace{3mm}Median [Min, Max]                 & 18 [0, 99]        & 18 [0, 99]        & 18 [0, 99]       \\
			\hspace{3mm}Missing                           & 683 (12\%)        & 644 (12\%)        & 39 (11\%)        \\
			\textbf{ED Systolic blood pressure}           &                   &                   &                  \\
			\hspace{3mm}Mean (SD)                         & 136 (28)          & 136 (28)          & 135 (31)         \\
			\hspace{3mm}Median [Min, Max]                 & 135 [0, 285]      & 135 [0, 285]      & 135 [0, 237]     \\
			\hspace{3mm}Missing                           & 660 (11\%)        & 626 (11\%)        & 34 (10\%)        \\
			\textbf{ED GCS}                               &                   &                   &                  \\
			\hspace{3mm}Mean (SD)                         & 14 (2)            & 14 (2)            & 14 (2)           \\
			\hspace{3mm}Median [Min, Max]                 & 15 [3, 15]        & 15 [3, 15]        & 15 [3, 15]       \\
			\hspace{3mm}Missing                           & 1008 (17\%)       & 950 (17\%)        & 58 (17\%)        \\
			\textbf{Time to first CT}                     &                   &                   &                  \\
			\hspace{3mm}Mean (SD)                         & 75 (137)          & 74 (137)          & 82 (139)         \\
			\hspace{3mm}Median [Min, Max]                 & 35 [0, 1415]      & 35 [0, 1415]      & 43 [6, 1339]     \\
			\hspace{3mm}Missing                           & 1206 (21\%)       & 1147 (21\%)       & 59 (17\%)        \\
			\textbf{Time to definitive treatment}         &                   &                   &                  \\
			\hspace{3mm}Mean (SD)                         & 263 (354)         & 260 (358)         & 278 (328)        \\
			\hspace{3mm}Median [Min, Max]                 & 110 [0, 2036]     & 104 [0, 2036]     & 154 [9, 1420]    \\
			\hspace{3mm}Missing                           & 4635 (80\%)       & 4453 (82\%)       & 182 (54\%)       \\
			\textbf{Intubated}                            &                   &                   &                  \\
			\hspace{3mm}No                                & 5285 (91\%)       & 4997 (92\%)       & 288 (85\%)       \\
			\hspace{3mm}Yes                               & 508 (9\%)         & 456 (8\%)         & 52 (15\%)        \\
			\hspace{3mm}Missing                           & 0 (0\%)           & 0 (0\%)           & 0 (0\%)          \\
			\textbf{Emergency procedure}                  &                   &                   &                  \\
			\hspace{3mm}Laparotomy                        & 114 (2\%)         & 100 (2\%)         & 14 (4\%)         \\
			\hspace{3mm}Craniotomy                        & 131 (2\%)         & 111 (2\%)         & 20 (6\%)         \\
			\hspace{3mm}Other                             & 787 (14\%)        & 699 (13\%)        & 88 (26\%)        \\
			\hspace{3mm}Radiological intervention         & 51 (\textless1\%) & 29 (\textless1\%) & 22 (6\%)         \\
			\hspace{3mm}Thoracotomy                       & 18 (\textless1\%) & 17 (\textless1\%) & 1 (\textless1\%) \\
			\hspace{3mm}Intracranial pressure measurement & 34 (\textless1\%) & 28 (\textless1\%) & 6 (2\%)          \\
			\hspace{3mm}Revascularisation                 & 22 (\textless1\%) & 15 (\textless1\%) & 7 (2\%)          \\
			\hspace{3mm}Pelvis packing                    & 2 (\textless1\%)  & 2 (\textless1\%)  & 0 (0\%)          \\
			\hspace{3mm}Missing                           & 4634 (80\%)       & 4452 (82\%)       & 182 (54\%)       \\
			\bottomrule
		\end{tabular}
	}
	\caption*{\small Time to first CT and Time to definitive treatment are measured in minutes from arrival at the hospital.\\
		\textit{Definition of abbreviations:} OFI = Opportunity for Improvement; ED = Emergency Department; GCS = Glascow Coma Scale.}
\end{table}

Selected patient characteristics for the data produced by the synthesising models for the first resample are presented
in \Cref{tab:tableonectgan,tab:tableonetvae} in the appendix. The \acrshort{tvae} dataset produced fewer \acrshort{ofi}
cases ($n=101$) than both the baseline and \acrshort{ctgan} ($n=500$). The overall mean age was 56 (\acrshort{sd}: 23)
and 51 (\acrshort{sd}: 26) for \acrshort{ctgan} and \acrshort{tvae}, respectively. The \acrshort{ctgan} dataset
produced more missing data, with the most missing data found in the Emergency procedure predictor with 87\% (76\% in
the \acrshort{tvae} group) and the least missing data with 0\% in the Intubated predictor (0\% in the \acrshort{tvae}
group). Excluding the Intubated predictor, the predictors with the lowest number of missing values were Age, Gender,
Dead at 30 days, Highest level of care, and Emergency department GCS with 36\% of data points missing values in the
\acrshort{ctgan} group. The same predictors in the \acrshort{tvae} group had all missing values in 6\% of data points.
The mean Emergency Department Glasgow Coma Scale was 21 and 20 for \acrshort{ctgan} and \acrshort{tvae}, respectively.
Additionally, the \acrshort{tvae} group did not produce ``Laparotomy'', ``Thoracotomy'', and ``Pelvis packing'' for
Emergency procedure.

\subsection{Model Performance}
The overall best data group was the baseline group with 6 out of 8 models scoring significantly their best
\acrshort{auc} score in that group. The $k$-nearest neighbour model scored its best \acrshort{auc} in the
\acrshort{ctgan} + baseline group. However, this performance improvement was insignificant with a mean \acrshort{auc}
of 0.717 (0.711, 0.722) in the \acrshort{ctgan} + baseline group and an \acrshort{auc} of 0.715 (0.71, 0.721) in the
baseline group.

The random forest model achieved the best overall \acrshort{auc} result out of all classification models in all data
groups, it scored an \acrshort{auc} of 0.792 (0.787, 0.797) in the baseline group, this was not significant compared to
the \acrshort{ctgan} + baseline group (\acrshort{auc}: 0.784 [0.779, 0.788]). The second best model, with an
overlapping \acrshort{ci} to the baseline random forest model, was CatBoost (\acrshort{auc}: 0.789 [0.784, 0.794]). The
worst performing model was the support vector machine, scoring its best \acrshort{auc} in the baseline group
(\acrshort{auc}: 0.675 [0.668, 0.681]). All synthesising models generated data that achieved significant discrimination
result (AUC $> 0.5$) apart from \acrshort{ctgan}, where the support vector machine could not achieve a significant
discriminatory \acrshort{auc} (\acrshort{auc}: 0.501 [0.489, 0.514]). Additionally, when adding the \acrshort{ctgan}
and baseline data, the support vector machine scored the worst overall score (\acrshort{auc}: 0.449 [0.424, 0.474]).
See \Cref{tab:modelperf} for \acrshort{auc} metrics for all classification models in each data group and corresponding
\acrshortpl{ci}.

\begin{table}
	\centering
	\renewcommand{\arraystretch}{0.8}
	\caption{\textbf{\acrshort{auc} performance in predicting \acrshort{ofi}.}
		The average \acrshort{auc} (95\% \acrshort{ci}) for classification models in each data group.
		The best performing data group for each classification model is displayed in bold.}%
	\label{tab:modelperf}
	\scalebox{0.73}{
		\begin{tabular}{lccccc}
			\toprule
			\textbf{Model}                       & \textbf{Baseline}       & \textbf{CTGAN} & \textbf{TVAE}  & \textbf{CTGAN + Baseline} & \textbf{TVAE + Baseline} \\
			\midrule
			\multirow{2}{*}{CatBoost}            & \textbf{0.789}          & 0.565          & 0.637          & 0.777                     & 0.776                    \\
			                                     & \textbf{(0.784, 0.794)} & (0.551, 0.579) & (0.615, 0.659) & (0.772, 0.782)            & (0.771, 0.781)           \\[0.5em]
			\multirow{2}{*}{Extra Trees}         & \textbf{0.776}          & 0.599          & 0.604          & 0.762                     & 0.765                    \\
			                                     & \textbf{(0.771, 0.781)} & (0.586, 0.611) & (0.585, 0.622) & (0.757, 0.768)            & (0.76, 0.77)             \\[0.5em]
			\multirow{2}{*}{k-NN}                & 0.715                   & 0.533          & 0.527          & \textbf{0.717}            & 0.689                    \\
			                                     & (0.71, 0.721)           & (0.52, 0.545)  & (0.521, 0.533) & \textbf{(0.711, 0.722)}   & (0.682, 0.695)           \\[0.5em]
			\multirow{2}{*}{LightGBM}            & \textbf{0.783}          & 0.549          & 0.604          & 0.772                     & 0.772                    \\
			                                     & \textbf{(0.778, 0.788)} & (0.537, 0.56)  & (0.587, 0.62)  & (0.767, 0.776)            & (0.766, 0.777)           \\[0.5em]
			\multirow{2}{*}{Logistic Regression} & \textbf{0.764}          & 0.595          & 0.618          & 0.748                     & 0.751                    \\
			                                     & \textbf{(0.759, 0.769)} & (0.58, 0.609)  & (0.592, 0.643) & (0.742, 0.754)            & (0.745, 0.756)           \\[0.5em]
			\multirow{2}{*}{Random forest}       & \textbf{0.792}          & 0.629          & 0.619          & 0.784                     & 0.777                    \\
			                                     & \textbf{(0.787, 0.797)} & (0.619, 0.639) & (0.603, 0.636) & (0.779, 0.788)            & (0.772, 0.783)           \\[0.5em]
			\multirow{2}{*}{SVM}                 & \textbf{0.675}          & 0.501          & 0.616          & 0.449                     & 0.648                    \\
			                                     & \textbf{(0.668, 0.681)} & (0.489, 0.514) & (0.593, 0.64)  & (0.424, 0.474)            & (0.641, 0.655)           \\[0.5em]
			\multirow{2}{*}{XGBoost}             & \textbf{0.771}          & 0.573          & 0.629          & 0.749                     & 0.745                    \\
			                                     & \textbf{(0.767, 0.776)} & (0.559, 0.588) & (0.602, 0.657) & (0.743, 0.754)            & (0.739, 0.751)           \\
			\bottomrule
		\end{tabular}

	}
	\caption*{\small \textit{Definition of abbreviations:}
		AUC = \acrlong{auc}; CI = \acrlong{ci}; k-NN = $k$-nearest neighbour; OFI = \acrlong{ofi};
		SVM = support vector machine.}
\end{table}

\begin{figure}
	\centering
	\includegraphics[width=0.85\textwidth]{figures/roc.pdf}
	\caption{\textbf{Average receiver operating characteristic curve for classification models within a data group.}
		Each data group curve was calculated by taking the average curve for all classification models within a
		data group. \\
		\textit{Definition of abbreviations:} AUC = \Acrlong{auc}; CTGAN = \Acrlong{ctgan}; TVAE = \Acrlong{tvae}.}%
	\label{fig:roc}
\end{figure}

The average receiver operating characteristic curve for classification models within a data group is presented in
\Cref{fig:roc}. The baseline, \acrshort{ctgan} + baseline, and \acrshort{tvae} + baseline data groups exhibit similar
receiver operating characteristic curves and \acrshort{auc} scores. The baseline group demonstrated the best overall
average \acrshort{auc} of 0.753. In comparison, the \acrshort{ctgan} data group demonstrated better performance in the
high false positive and true positive rate range than the \acrshort{tvae} data group, but lower performance in the low
false positive and true positive rate range. This was also seen in the curve when the baseline data was added to each
group. The \acrshort{ctgan} data group performed the worst, with an average \acrshort{auc} of 0.555.

\section{Discussion}
The aim of this study was to evaluate the effectiveness of generating synthetic data in a trauma setting using a
\acrshort{gan} and \acrshort{vae} in improving \acrshort{ofi} prediction models. The performance of classification
models trained with the generated data from the \acrshort{gan} and \acrshort{vae} was significantly worse in predicting
\acrshortpl{ofi} compared to models trained with the baseline data. On average, the 8 classifications models tested
performed 0.198 and 0.187 \acrshort{auc} points worse when trained on synthetic data generated by the \acrshort{gan}
and the \acrshort{vae} respectively, compared to being trained on the baseline data. The results display the current
infeasibility of generating synthetic data to improve \acrshort{ofi} prediction models in a trauma setting.

The \acrshort{vae} based model implemented in this study showed an overall better performance in generating data than
the \acrshort{gan} based model. Four out of eight classification models achieved significantly better performance using
the \acrshort{vae} data compared to the \acrshort{gan} data, the remaining four models were non-inferior. However, the
average difference in \acrshort{auc} for the classification models trained with \acrshort{vae}-generated data versus
\acrshort{gan}-generated data was minimal. Specifically, the average \acrshort{auc} for models trained with
\acrshort{vae}-generated data was 0.566, while for models trained with \acrshort{gan}-generated data it was 0.555. This
may indicate that generating synthetic data in this context is a challenging task, rather than being attributed to the
specific methods used.

The quality of the data generated by the various predictors for the synthetic datasets varies. For example, both
synthesisers erroneously produced Glasgow coma scale scores over 15. The \acrshort{tvae} model's dataset has overall
means and prevalences that are closer to those of the baseline dataset. However, the \acrshort{tvae} model seems to
produce a less varied \acrshort{ofi} positive group. As shown in \Cref{tab:tableonetvae}, \acrshort{tvae} for
\acrshort{ofi} positive cases ($n=101$) produced 100\% male, 95\% not dead at 30 days, and 94\% with the intensive care
unit as the highest level of care. Compared with the baseline characteristics in \Cref{tab:tableone}, the prevalence
for the same predictors were 64\%, 84\%, and 30\%, respectively. When it comes to categorical predictors, the
\acrshort{ctgan} model's data was more aligned with the baseline characteristics, however, included many more missing
variables then both baseline and the \acrshort{tvae} data. In conclusion, both synthetic data generation methods
produced low-quality data. The conclusion is supported by the variable distribution and poor performance of
classification models trained on both datasets.

An interesting observation is that when combining the \acrshort{vae} or \acrshort{gan} data with the baseline data, the
performance difference from the baseline data, though significant, was minimal. The average \acrshort{auc} difference
of the three best performing classification models (CatBoost, LightGBM, random forest) baseline performance versus the
\acrshort{gan} + baseline performance was 0.01. For \acrshort{vae}, it was also 0.01. Despite the baseline and
synthetic datasets being equally large, the classification models were barely affected by the poor quality synthetic
data. The test data used in each resample split was not presented to the models until evaluation, meaning that the
classification models learnt the most important features in the combined data despite the noise added by the synthetic
data.

The results that a combination of synthetic and baseline data perform nearly as well as solely baseline data hints that
an improvement of the synthetic data quality might improve the performance of classification models trained on both
baseline + synthetic data compared to only baseline. Despite the synthetic data not inherently improving the
performance of the classification models on its own, the observation that an improvement in the quality of synthetic
data may lead to better performance when combined with baseline data suggests that the synthetic data lacks data points
close to those in the baseline data and instead focuses on edge cases. As a result, models trained on only synthetic
data may perform poorly due to learning only edge cases, whereas models trained on a combination of baseline and
synthetic data may result in more robust models, as they learn from both the bulk data and edge cases.

The baseline results exhibit similarities to those reported by Attergrim et al.~\cite{attergrim_predicting_2023}, who
found that LightGBM and random forest were the best-performing models. Hernandez et al.~\cite{hernandez_synthetic_2022}
reported varied performance of synthetic data generation using \acrlongpl{gan} in different medical contexts. One
conclusion drawn by Hernandez and colleagues was that the existing \acrlong{gan}-based approaches are not generalisable
to all types of tabular data. This is further supported by the present study where the tested \acrshort{ctgan}
displayed poor performance compared to the baseline. Abedi et al.~\cite{abedi_gan-based_2022} investigated how the size
of the training data affected the performance of classification models trained with synthetic data. For certain
training data sizes, the results were similar to those found in the current study. This may indicate that, for the time
being, there is not enough data to use the methods tested.

Finally, despite the results of the current study, if it was possible to generate synthetic data to improve
\acrshort{ml} prediction model performance, it would most likely be better to use the synthesising model directly as a
prediction model. Synthetic data generators cannot add more information than what was already in the baseline data;
they learn features and correlations in the baseline data and use them to generate data. Similarly, classification
models learn features and correlations in the baseline data but use them to predict a class instead. Therefore, to
produce quality data that a classification model can learn from, the synthetic data generator has to learn features and
correlations in the baseline dataset better than what the classification models can. If we were to change the synthetic
data generator's task from generating data to classifying data, we could expect the same, if not better, performance
than the previously used classification model. This has been explored with using \acrshortpl{gan} for classification by
Israel et al.~\cite{israel_generative_2017}.

\subsection{Strengths and Limitations}
There are several strengths to the present study. Firstly, the code used to develop the models was developed without
running it on the true data until validation time. Secondly, the methodology uses a robust way of validation.
Resampling using 100 resamples allows for virtual simulation of rerunning the experiment 100 times, providing the
ability to calculate confidence intervals to ensure statistical significance between models. Thirdly, the
classification models are based on an already validated methodology using state-of-the-art models. This means that we
can compare our baseline results to previous results to ensure that the current models have been applied correctly.

However, there are limitations to the study. Firstly, the sample size is relatively small ($n = 5793$). This can be
compared with the original implementation studies that use at least tens of thousands of data points, and in many
cases, much more~\cite{xu_modeling_2019,karras_training_2020}. Furthermore, only two synthetic data generation models
were tested, whereas there exist many more types of models, some more specific for medical
data~\cite{hernandez_synthetic_2022}.

\subsection{Practical Applications}
The present study, despite the inferior results of the tested synthetic data generation models, has developed a
framework that enables the validation of synthetic data generation methods using machine learning models for predicting
\acrshort{ofi}. This modular framework facilitates the testing and validation of further synthesising models with
minimal effort. Moreover, the framework is extensible in terms of the classification models, allowing for the addition
of more.

The implication of this development is that synthetic data generation methods can be rapidly tested and validated. If,
or when, a synthetic data generation method is validated that can generate accurate synthetic trauma datasets, these
could theoretically be distributed to the wider scientific community. This could have a significant impact,
particularly in furthering the possibility of open and reproducible science in medicine. It could also lead to easier
distribution of data, facilitating simpler methods for external validation. Finally, it allows for the possibility that
several research groups can work on the same problem.

However, ethical questions have to be considered. For instance, whether the synthetic dataset requires the same ethical
considerations as the true data. Another issue is ensuring that the synthetic data represents the population as a
whole. As evidenced by the TVAE-generated data, the OFI positive population was predominantly male. This could result
in gender, age, or other socio-demographic factors being over or under-represented in the synthetic datasets. It is
crucial to ensure that future synthetic datasets do not suffer from this issue to ensure health equity.

\subsection{Future Studies}
Future studies should aim to investigate whether other synthetic data generation techniques can provide beneficial
results, building on the framework developed in the present study. This could help determine whether synthetic data
generation is possible for smaller trauma datasets. One promising technique worth exploring is diffusion probabilistic
models, which have gained popularity in text-to-image models~\cite{rombach_high-resolution_2022,
	ramesh_hierarchical_2022}. Diffusion models have already demonstrated promising results in surpassing traditional GAN
and VAE-based models in generating tabular data~\cite{kotelnikov_tabddpm_2022}.

Furthermore, it would be valuable to investigate whether machine learning models trained on both baseline and synthetic
data exhibit improved transferability to other sites compared to models trained solely on baseline data. As the present
study focused on a single dataset from a specific site (Karolinska University Hospital, Solna, Sweden), it would be
interesting to explore if the inclusion of synthetic data, which can potentially augment the diversity and
representativeness of the training data, could enhance the models' ability to generalize across different sites. By
testing the models' performance on multiple datasets from different sites, the potential for increased transferability
of models trained with synthetic data could be assessed. This could provide insights into the robustness and
generalizability of the models, which has important implications for real-world applications where models are deployed
across different environments.

In addition, it is important to implement supplementary methods for evaluating the quality and privacy of the generated
datasets. The current method only tests quality in terms of \acrshort{ofi} prediction, but there are many other
correlations that need to be assessed. Furthermore, the actual performance of generating anonymised datasets was not
tested. Both these questions could be explored by implementing the multifaceted benchmarking framework for synthetic
electronic health records introduced by Yan et al.~\cite{yan_multifaceted_2022}.

\section{Conclusions}
The present study suggests that the current \acrshort{gan} and \acrshort{vae}-based methods are unsuitable for
generating synthetic data for \acrshort{ofi} prediction in trauma patient care. The performance of the \acrshort{ml}
models trained on data synthesised by the \acrshort{ctgan} and \acrshort{tvae} models in predicting \acrshortpl{ofi} in
trauma patient care was significantly worse then models trained on baseline data. However, further research is required
to determine whether other \acrshort{gan}-based models, with additional adjustments, could produce an acceptable
performance.

\newpage

\singlespacing

\printbibliography

\newpage

\begin{appendices}
	\setcounter{table}{0}
	\renewcommand{\thetable}{A\arabic{table}}

	\setcounter{figure}{0}
	\renewcommand{\thefigure}{A\arabic{figure}}

	\setcounter{secnumdepth}{1}

	\section{Tables}
	\begin{table}[h]
		\centering
		\renewcommand{\arraystretch}{1.2}

		\caption{\textbf{All the currently used audit filters at Karolinska University Hospital.}}
		\label{tab:auditfilters}

		\begin{tabular}{@{}|p{0.85\linewidth}|@{}}
			\hline
			\multicolumn{1}{|c|}{\textbf{Audit filter}}                                                \\\hline
			Systolic blood pressure less than 90                                                       \\\hline
			Glasgow coma scale less than 9 and not intubated                                           \\\hline
			Injury severity score greater than 15 but not admitted to the intensive care unit          \\\hline
			Time to acute intervention more than 60 minutes from arrival to hospital                   \\\hline
			Time to computed tomography more than 30 minutes from arrival to hospital                  \\\hline
			No anticoagulant therapy within 72 hours after traumatic brain injury                      \\\hline
			The presence of cardio-pulmonary resuscitation with thoracotomy                            \\\hline
			The presence of a liver or spleen injury                                                   \\\hline
			Massive transfusion, defined as 10 or more units of packed red blood cells within 24 hours \\\hline
			Other non-defined reason found by first reviewing nurse                                    \\\hline
		\end{tabular}
	\end{table}

	\renewcommand*{\arraystretch}{1.2}
	\begin{longtable}[c]{@{}|l|p{0.55\linewidth}|@{}}
		\caption{\textbf{Predictors used in the development of the \acrlong{ofi} classification and synthesising models.}}%
		\label{tab:predictors}                                                                   \\
		\hline
		\multicolumn{1}{|c|}{\textbf{SweTrau code}} & \multicolumn{1}{|c|}{\textbf{Description}} \\\hline
		\endfirsthead
		%
		\endhead
		%
		AlarmRePrioritised                          & Reprioritisation of trauma code            \\\hline FirstTraumaDT\_NotDone & Trauma CT not done \\\hline ISS
		                                            & Injury Severity Score                      \\\hline NumberOfActions & Number of Actions done \\\hline NumberOfInjuries & Number of
		injuries                                                                                 \\\hline TraumaAlarmAtHospital & Type of trauma code criteria \\\hline TraumaAlarmCriteria & Trauma code at
		hospital                                                                                 \\\hline dt\_alarm\_hosp & DT alarm to ED \\\hline dt\_alarm\_scene & DT alarm to arrival at scene \\\hline
		dt\_ed\_emerg\_proc                         & DT arrival at ED to emergency procedure    \\\hline dt\_ed\_first\_ct & DT arrival at ED to first CT
		\\\hline dt\_ed\_norm\_be & DT arrival ED to normalised base excess \\\hline ed\_be\_art & First base excess at the ED
		\\\hline ed\_be\_art\_NotDone & Base excess not taken at ED \\\hline ed\_emerg\_proc & Emergency procedure at ED
		\\\hline ed\_emerg\_proc\_other & Other emergency procedure at ED \\\hline ed\_gcs\_motor & Motor response according to
		GCS at ED                                                                                \\\hline ed\_gcs\_sum & GCS score at ED \\\hline ed\_inr & First Prothrombin time (international normalise
		ratio)                                                                                   \\\hline ed\_inr\_NotDone & Prothrombin time (international normalised ratio) not taken at ED \\\hline
		ed\_intub\_type                             & Intubation type at ED                      \\\hline ed\_intubated & Was intubated at ED \\\hline ed\_rr\_value &
		Respiratory rate at ED                                                                   \\\hline ed\_sbp\_value & Systolic blood pressure at ED \\\hline ed\_tta & Trauma team activated
		\\\hline hosp\_dischg\_dest & Discharge destination \\\hline hosp\_los\_days & Total admittance days at hospital
		\\\hline hosp\_vent\_days & Total amount of days in a ventilator \\\hline host\_care\_level & Highest level of care
		\\\hline host\_transfered & Transferred from/to another hospital \\\hline host\_vent\_days\_NotDone & No days spent on
		a ventilator                                                                             \\\hline inj\_dominant & Dominant type of injury \\\hline inj\_intention & Injury intention \\\hline
		inj\_mechanism                              & Dominant injury mechanism                  \\\hline iva\_dagar\_n & Total amount of days in the intensive care unit
		\\\hline iva\_vardtillfallen\_n & Amount of times admitted to the intensive care unit \\\hline pre\_card\_arrest & PH
		cardiac arrest                                                                           \\\hline pre\_gcs\_motor & PH motor response according to GCS \\\hline pre\_gcs\_sum & PH GCS score
		\\\hline pre\_intub\_type & Type of intubation PH \\\hline pre\_intubated & Was intubated PH \\\hline pre\_provided &
		Level of care given PH                                                                   \\\hline pre\_rr\_value & PH Respiratory rate \\\hline pre\_sbp\_value & PH Systolic blood
		pressure                                                                                 \\\hline pre\_transport & Transport type PH \\\hline pt\_Gender & Gender \\\hline pt\_age\_yrs & Patient age
		\\\hline pt\_asa\_preinjury & American Society of Anesthesiologists Class pre-injury \\\hline res\_gos\_dischg &
		Glascow outcome score at discharge                                                       \\\hline res\_survival & Mortality \\\hline \caption*{\small \textit{Definition of
				abbreviations:} CT = Computer Tomography; DT = Delta Time; ED = Emergency department; GCS = Glascow Coma Scale; PH =
			Pre-hospital.}%
	\end{longtable}

	\begin{table}[t!]
		\centering
		\renewcommand{\arraystretch}{0.9}
		\caption{\textbf{Selected patient demographic and clinical characteristics}. Subdivided by \acrshort{ofi} outcome for \acrshort{ctgan} data.}
		\label{tab:tableonectgan}
		\scalebox{0.7}{
			\begin{tabular}{lccc}
				\toprule
				                                              & \textbf{Overall}  & \textbf{No OFI}   & \textbf{OFI}     \\
				\midrule
				                                              & $n=5793$          & $n=5293$          & $n=500$          \\
				\textbf{Age}                                  &                   &                   &                  \\
				\hspace{3mm}Mean (SD)                         & 56 (23)           & 56 (23)           & 55 (23)          \\
				\hspace{3mm}Median [Min, Max]                 & 56 [15, 100]      & 56 [15, 100]      & 53 [15, 100]     \\
				\hspace{3mm}Missing                           & 2112 (36\%)       & 1847 (35\%)       & 265 (53\%)       \\
				\textbf{Gender}                               &                   &                   &                  \\
				\hspace{3mm}Female                            & 1566 (27\%)       & 1465 (28\%)       & 101 (20\%)       \\
				\hspace{3mm}Male                              & 2122 (37\%)       & 1988 (38\%)       & 134 (27\%)       \\
				\hspace{3mm}Missing                           & 2105 (36\%)       & 1840 (35\%)       & 265 (53\%)       \\
				\textbf{Dead at 30 days}                      &                   &                   &                  \\
				\hspace{3mm}No                                & 3577 (62\%)       & 3350 (63\%)       & 227 (45\%)       \\
				\hspace{3mm}Yes                               & 121 (2\%)         & 113 (2\%)         & 8 (2\%)          \\
				\hspace{3mm}Missing                           & 2095 (36\%)       & 1830 (35\%)       & 265 (53\%)       \\
				\textbf{Highest level of care}                &                   &                   &                  \\
				\hspace{3mm}General ward                      & 2041 (35\%)       & 1928 (36\%)       & 113 (23\%)       \\
				\hspace{3mm}Intensive care unit               & 842 (15\%)        & 780 (15\%)        & 62 (12\%)        \\
				\hspace{3mm}ED                                & 410 (7\%)         & 385 (7\%)         & 25 (5\%)         \\
				\hspace{3mm}Surgical ward                     & 334 (6\%)         & 305 (6\%)         & 29 (6\%)         \\
				\hspace{3mm}Specialist/Intermediate ward      & 67 (1\%)          & 62 (1\%)          & 5 (1\%)          \\
				\hspace{3mm}Missing                           & 2099 (36\%)       & 1833 (35\%)       & 266 (53\%)       \\
				\textbf{Injury severity score}                &                   &                   &                  \\
				\hspace{3mm}Mean (SD)                         & 7 (9)             & 7 (9)             & 8 (9)            \\
				\hspace{3mm}Median [Min, Max]                 & 5 [0, 75]         & 5 [0, 75]         & 8 [0, 74]        \\
				\hspace{3mm}Missing                           & 2116 (37\%)       & 1851 (35\%)       & 265 (53\%)       \\
				\textbf{ED Respiratory Rate}                  &                   &                   &                  \\
				\hspace{3mm}Mean (SD)                         & 27 (24)           & 27 (23)           & 28 (24)          \\
				\hspace{3mm}Median [Min, Max]                 & 21 [4, 99]        & 21 [4, 99]        & 22 [6, 99]       \\
				\hspace{3mm}Missing                           & 2311 (40\%)       & 2025 (38\%)       & 286 (57\%)       \\
				\textbf{ED Systolic blood pressure}           &                   &                   &                  \\
				\hspace{3mm}Mean (SD)                         & 148 (36)          & 148 (36)          & 141 (35)         \\
				\hspace{3mm}Median [Min, Max]                 & 143 [0, 285]      & 143 [0, 285]      & 138 [0, 274]     \\
				\hspace{3mm}Missing                           & 2238 (39\%)       & 1960 (37\%)       & 278 (56\%)       \\
				\textbf{ED GCS}                               &                   &                   &                  \\
				\hspace{3mm}Mean (SD)                         & 21 (32)           & 21 (32)           & 22 (24)          \\
				\hspace{3mm}Median [Min, Max]                 & 15 [3, 999]       & 15 [3, 999]       & 15 [3, 121]      \\
				\hspace{3mm}Missing                           & 2102 (36\%)       & 1839 (35\%)       & 263 (53\%)       \\
				\textbf{Time to first CT}                     &                   &                   &                  \\
				\hspace{3mm}Mean (SD)                         & 51 (80)           & 50 (79)           & 58 (91)          \\
				\hspace{3mm}Median [Min, Max]                 & 29 [0, 1033]      & 29 [0, 1033]      & 34 [2, 750]      \\
				\hspace{3mm}Missing                           & 2477 (43\%)       & 2167 (41\%)       & 310 (62\%)       \\
				\textbf{Time to definitive treatment}         &                   &                   &                  \\
				\hspace{3mm}Mean (SD)                         & 314 (294)         & 317 (298)         & 285 (245)        \\
				\hspace{3mm}Median [Min, Max]                 & 267 [1, 1727]     & 267 [1, 1727]     & 266 [18, 1311]   \\
				\hspace{3mm}Missing                           & 4925 (85\%)       & 4490 (85\%)       & 435 (87\%)       \\
				\textbf{Intubated}                            &                   &                   &                  \\
				\hspace{3mm}No                                & 5490 (95\%)       & 5025 (95\%)       & 465 (93\%)       \\
				\hspace{3mm}Yes                               & 303 (5\%)         & 268 (5\%)         & 35 (7\%)         \\
				\hspace{3mm}Missing                           & 0 (0\%)           & 0 (0\%)           & 0 (0\%)          \\
				\textbf{Emergency procedure}                  &                   &                   &                  \\
				\hspace{3mm}Laparotomy                        & 242 (4\%)         & 225 (4\%)         & 17 (3\%)         \\
				\hspace{3mm}Thoracotomy                       & 134 (2\%)         & 117 (2\%)         & 17 (3\%)         \\
				\hspace{3mm}Other                             & 32 (\textless1\%) & 32 (\textless1\%) & 0 (0\%)          \\
				\hspace{3mm}Pelvis packing                    & 117 (2\%)         & 111 (2\%)         & 6 (1\%)          \\
				\hspace{3mm}Radiological intervention         & 68 (1\%)          & 66 (1\%)          & 2 (\textless1\%) \\
				\hspace{3mm}Intracranial pressure measurement & 48 (\textless1\%) & 44 (\textless1\%) & 4 (\textless1\%) \\
				\hspace{3mm}Craniotomy                        & 44 (\textless1\%) & 39 (\textless1\%) & 5 (1\%)          \\
				\hspace{3mm}Revascularisation                 & 83 (1\%)          & 79 (1\%)          & 4 (\textless1\%) \\
				\hspace{3mm}Missing                           & 5025 (87\%)       & 4580 (87\%)       & 445 (89\%)       \\
				\bottomrule
			\end{tabular}
		}
		\caption*{\small Time to first CT and Time to definitive treatment are measured in minutes from arrival at the hospital.\\
			\textit{Definition of abbreviations:} OFI = Opportunity for Improvement; ED = Emergency Department; GCS = Glascow Coma Scale.}
	\end{table}

	\begin{table}[t!]
		\centering
		\renewcommand{\arraystretch}{0.9}
		\caption{\textbf{Selected patient demographic and clinical characteristics}. Subdivided by \acrshort{ofi} outcome for \acrshort{tvae} data.}
		\label{tab:tableonetvae}
		\scalebox{0.7}{
			\begin{tabular}{lccc}
				\toprule
				                                              & \textbf{Overall}  & \textbf{No OFI}   & \textbf{OFI}  \\
				\midrule
				                                              & $n=5894$          & $n=5793$          & $n=101$       \\
				\textbf{Age}                                  &                   &                   &               \\
				\hspace{3mm}Mean (SD)                         & 51 (26)           & 51 (26)           & 50 (20)       \\
				\hspace{3mm}Median [Min, Max]                 & 47 [15, 100]      & 47 [15, 100]      & 47 [15, 92]   \\
				\hspace{3mm}Missing                           & 360 (6\%)         & 360 (6\%)         & 0 (0\%)       \\
				\textbf{Gender}                               &                   &                   &               \\
				\hspace{3mm}Male                              & 4808 (82\%)       & 4707 (81\%)       & 101 (100\%)   \\
				\hspace{3mm}Female                            & 727 (12\%)        & 727 (13\%)        & 0 (0\%)       \\
				\hspace{3mm}Missing                           & 359 (6\%)         & 359 (6\%)         & 0 (0\%)       \\
				\textbf{Dead at 30 days}                      &                   &                   &               \\
				\hspace{3mm}No                                & 5224 (89\%)       & 5128 (89\%)       & 96 (95\%)     \\
				\hspace{3mm}Yes                               & 311 (5\%)         & 306 (5\%)         & 5 (5\%)       \\
				\hspace{3mm}Missing                           & 359 (6\%)         & 359 (6\%)         & 0 (0\%)       \\
				\textbf{Highest level of care}                &                   &                   &               \\
				\hspace{3mm}Intensive care unit               & 1416 (24\%)       & 1321 (23\%)       & 95 (94\%)     \\
				\hspace{3mm}General ward                      & 2099 (36\%)       & 2099 (36\%)       & 0 (0\%)       \\
				\hspace{3mm}Surgical ward                     & 1120 (19\%)       & 1114 (19\%)       & 6 (6\%)       \\
				\hspace{3mm}ED                                & 883 (15\%)        & 883 (15\%)        & 0 (0\%)       \\
				\hspace{3mm}Specialist/Intermediate ward      & 14 (\textless1\%) & 14 (\textless1\%) & 0 (0\%)       \\
				\hspace{3mm}Missing                           & 362 (6\%)         & 362 (6\%)         & 0 (0\%)       \\
				\textbf{Injury severity score}                &                   &                   &               \\
				\hspace{3mm}Mean (SD)                         & 11 (10)           & 10 (10)           & 32 (7)        \\
				\hspace{3mm}Median [Min, Max]                 & 9 [0, 71]         & 9 [0, 71]         & 32 [10, 45]   \\
				\hspace{3mm}Missing                           & 366 (6\%)         & 366 (6\%)         & 0 (0\%)       \\
				\textbf{ED Respiratory Rate}                  &                   &                   &               \\
				\hspace{3mm}Mean (SD)                         & 24 (20)           & 24 (21)           & 22 (3)        \\
				\hspace{3mm}Median [Min, Max]                 & 18 [5, 99]        & 18 [5, 99]        & 23 [14, 25]   \\
				\hspace{3mm}Missing                           & 896 (15\%)        & 878 (15\%)        & 18 (18\%)     \\
				\textbf{ED Systolic blood pressure}           &                   &                   &               \\
				\hspace{3mm}Mean (SD)                         & 139 (21)          & 139 (21)          & 125 (31)      \\
				\hspace{3mm}Median [Min, Max]                 & 143 [0, 222]      & 143 [0, 222]      & 129 [54, 173] \\
				\hspace{3mm}Missing                           & 451 (8\%)         & 447 (8\%)         & 4 (4\%)       \\
				\textbf{ED GCS}                               &                   &                   &               \\
				\hspace{3mm}Mean (SD)                         & 20 (20)           & 20 (20)           & 14 (2)        \\
				\hspace{3mm}Median [Min, Max]                 & 15 [3, 105]       & 15 [3, 105]       & 14 [5, 16]    \\
				\hspace{3mm}Missing                           & 362 (6\%)         & 362 (6\%)         & 0 (0\%)       \\
				\textbf{Time to first CT}                     &                   &                   &               \\
				\hspace{3mm}Mean (SD)                         & 45 (32)           & 45 (31)           & 54 (48)       \\
				\hspace{3mm}Median [Min, Max]                 & 33 [0, 330]       & 32 [0, 330]       & 39 [15, 282]  \\
				\hspace{3mm}Missing                           & 1050 (18\%)       & 1050 (18\%)       & 0 (0\%)       \\
				\textbf{Time to definitive treatment}         &                   &                   &               \\
				\hspace{3mm}Mean (SD)                         & 244 (343)         & 256 (351)         & 67 (20)       \\
				\hspace{3mm}Median [Min, Max]                 & 84 [0, 1393]      & 87 [0, 1393]      & 65 [27, 119]  \\
				\hspace{3mm}Missing                           & 4294 (73\%)       & 4294 (74\%)       & 0 (0\%)       \\
				\textbf{Intubated}                            &                   &                   &               \\
				\hspace{3mm}No                                & 5269 (89\%)       & 5264 (91\%)       & 5 (5\%)       \\
				\hspace{3mm}Yes                               & 625 (11\%)        & 529 (9\%)         & 96 (95\%)     \\
				\hspace{3mm}Missing                           & 0 (0\%)           & 0 (0\%)           & 0 (0\%)       \\
				\textbf{Emergency procedure}                  &                   &                   &               \\
				\hspace{3mm}Other                             & 462 (8\%)         & 430 (7\%)         & 32 (32\%)     \\
				\hspace{3mm}Intracranial pressure measurement & 606 (10\%)        & 584 (10\%)        & 22 (22\%)     \\
				\hspace{3mm}Radiological intervention         & 52 (\textless1\%) & 52 (\textless1\%) & 0 (0\%)       \\
				\hspace{3mm}Craniotomy                        & 309 (5\%)         & 299 (5\%)         & 10 (10\%)     \\
				\hspace{3mm}Revascularisation                 & 6 (\textless1\%)  & 6 (\textless1\%)  & 0 (0\%)       \\
				\hspace{3mm}Missing                           & 4459 (76\%)       & 4422 (76\%)       & 37 (37\%)     \\
				\bottomrule
			\end{tabular}
		}
		\caption*{\small Time to first CT and Time to definitive treatment are measured in minutes from arrival at the hospital.\\
			\textit{Definition of abbreviations:} OFI = Opportunity for Improvement; ED = Emergency Department; GCS = Glascow Coma Scale.}
	\end{table}
\end{appendices}

\end{document}
